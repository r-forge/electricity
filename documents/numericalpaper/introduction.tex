\section{Introduction}

With the ongoing deregulation of electricity markets, firms behave differently in this new setting. When investing in generation capacity, they are faced with uncertainties related to the future demand of electricity and to the investment and production decisions of their competitors. \cite{Ventosa2005} give an overview about decision support models used in electricity market modeling. There exist two types of equilibrium models. First, Cournot competition where the players compete in quantity and second, supply function equilibria (SFE) where the firms compete over their offer curves. What both approaches have in common, is that they are based on the concept of \cite{Nash1951} equilibria, i.e. when each players strategy is the best response to its opponents actions, the market has found an equilibrium.

An important aspect of a dynamic oligopoly model is the assumption about the information structure. The literature distinguishes between the following three cases \citep[see, e.g.,][]{Cellini2004}. \emph{Open-loop} equilibria only take into account the initial state variables and the time, but they do not include any strategic interaction based on the evolution of the state variables over time. They are not subgame perfect, meaning that not in all stages the strategies have to be Nash equilibria, and players must commit to their decisions forever. The initial and current levels of all state variables are taken into account in \emph{closed-loop} equilibria, which can only be found for special cases \citep[see][]{Murphy2005}. In a \emph{feedback} equilibrium information about the accumulated stock of each state variable at the current date is included.

There is a wide diversity of papers that deals with Cournot equilibria in the spot market for electricity \citep[see, e.g.][]{Borenstein1999, Otero-Novas2000}. Here, the strategic decision making is performed under a short time horizon. The following recent papers specifically model the investment problem in electricity markets, which clearly has a medium to long-term focus.

\cite{Pineau2003} formulate a model with sample path optimization for the Finnish electricity market. The information structure is open-loop and the strategies of the players adapt to the sample path of the stochastic variable, i.e. demand. They consider an open-loop information structure appropriate because electricity generators stick to their investment decisions for some time while only adjusting for shocks in exogenous variables. In their results, the presence of uncertain demand growth and strategic behavior can reduce investment incentives. They solve the optimization problem with two methods, i.e. the first directly solves the necessary conditions of the Nash equilibrium, the second method uses variational inequalities. \cite{Murphy2005} compare an open-loop and a closed-loop model to the perfect competition case. They provide proofs for the existence and the uniqueness of the solutions. They convert the inverse of the load duration curve to a probability density function and interpret it as a way to cope with the overall uncertainty about the future demand for electricity. The closed-loop formulation leads to a two-stage equilibrium problem which is difficult to handle numerically. The search through second-stage equilibria needs to be done by enumerating all complementarity sets of the second-stage problem.
\cite{Neuhoff2005} analyze electricity markets with transmission constraints but their models are static. \cite{Sauma2006} evaluate social welfare implications of investments in transmission capacity. Market deregulation has led to independent companies that operate the transmission network but may not necessarily work in the generation business as well. For the generators action they use a simplified version of the \cite{Murphy2005} generation capacity investment model. \cite{Centeno2007} model medium-term strategic generation decisions based on a conjectured price-response market equilibrium. \cite{Lise2008} propose a model that consists of a static part that reflects trade, capacity and environmental constraints, and a dynamic part which focuses on investment decisions in the long run. 

%Das geh�rt den den Literature Review An alternative setup to ours is presented in \cite{Boom2007}. Here, a blackout can occur in the case of which all players loose all their rents. \cite{Chao2004} show how market power is mitigated and how new investments are facilitated if option contracts are introduced. We do not account for such market features, although it might be an interesting alternative for further research.
%Capacity investments have also been studied from a real option perspective which incorporates the value to delay investments by \cite{Roques2006}.

\cite{Genc2007} consider several stochastic programming formulations of the dynamic oligopolistic investment game. Their games with expected scenarios (GES) are similar to the formulation in \cite{Pineau2003}. They consider the so-called games with probabilistic scenarios (GPS) as an appropriate model, which they test with data for the Canadian electricity market. One of their conclusions is that stochastic programming models can cope with game formulations, that can not be addressed with dynamic programming. In \cite{Genc2008} a more detailed example for the Ontario electricity market can be found, which is again formulated as a game with probabilistic scenarios. 

The contribution of this paper is that it extends the work of \cite{Genc2007} and  \cite{Genc2008} in several important directions. First, we take the model to a more detailed level by using all available technologies and a realistic approximation of the load duration curve. Second, we formulated the mixed complementarity problem with the compact formulation instead of the split-variable formulation, which has the main advantage that it produces less variables and equations. Finally, the present paper is - as far as we know - the first application of a stochastic mixed complementarity problem to the German electricity market.

In section \ref{sec:model} we present the model and show how to obtain the MCP formulation. Section \ref{sec:germ-electr-mark} gives an overview about the German electricity market. In section \ref{sec:results} the model solution and some sensitivity tests are presented. Section \ref{sec:conclusion} conludes.


%%% Local Variables: 
%%% mode: latex
%%% TeX-master: "gencapinvest"
%%% End: 
