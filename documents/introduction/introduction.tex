\section{Introduction}

With the ongoing deregulation of electricity markets firms behave different in this new market setting. When investing in new generation capacity firms are faced with uncertainties related to the future demand of electricity and to the investment and production decisions of their competitors. \cite{Ventosa2005} gives an overview about decision support models used in electricity market modeling. There exist two types of equilibrium models. First, Cournout competition where the players compete in quantity and second, supply function equilibria (SFE) where the firms compete over their offer curves. What both approaches have in common is that they are based on the concept of \cite{Nash1951} equilibra, i.e. when each players strategy is the best response to its opponents actions the market has found an equilibrium. In the following, we will focus on the Cournot approach which is easier applicable to real world problems, because SFE can only be found under strong assumptions and convergence problems arise quite often.

An important aspect of a dynamic oligopoly model is the assumption about the information structure. The literature disinguishes between the following three cases \citep[see, e.g.,][]{Cellini2004}. \emph{Open-loop} equilibria only take into account the inital state variables and the time, but they do not include any strategic interaction based on the evolution of the state variables over time. They are not subgame perfect, meaning that not in all stages the strategies have to be Nash equilibria, and players must commit to their decision forever. The initial and current levels of all state variables are taken into account in a \emph{closed-loop} equilibrium. They can only be found for special cases. \cite{Murphy2005} consider a two-stage game where they interpret the close-loop game as electricity industry with a spot market, where players make an investment decision in the first stage and know how they will play against each other when producing the electricity. In a \emph{feedback} equilibrium information about the accumulated stock of each state variable at the current date is included.

There is a wide diversity of papers that deals with Cournot equilibria in the spot market for electricity \citep[see, e.g.][]{Borenstein1999, Otero-Novas2000}. Here, the strategic decision making is performed under a short time horizon. The following recent papers specifically model the investment problem in electricty markets, which clearly has a medium to long-term focus.

\cite{Chuang2001}

\cite{Ventosa2002}

\cite{Hogendorn2003}

\cite{Chaton2003}

\cite{Pineau2003} formulate a model with sample path optimization for the Finnish electricity market.The information structure is open-loop and the strategies of the players adapt to the sample path of the stochastic variable, i.e. demand. They consider an open-loop information structure appropiate because electricity generators stick to their investment decisions for some time while only adjusting for shocks in exogenous variables. In their results the presence of uncertain demand growth and strategic behaviour can reduce investment incentives.

\cite{Murphy2005} compare an open-loop and a closed-loop model to the perfect competition case. They provide proofs for the existence and the uniqueness of the solutions. They convert the inverse of the load duration curve to a probability density function and interpret it as a way to cope with the overall uncertainty about the future demand for electricity. The closed-loop formulation leads to a two-stage equilibrium problem which is difficult to handle numerically. The search through second-stage equilibria needs to be done by enumerating all complementarity sets of the second-stage problem.

\cite{Neuhoff2005} analyze electricty markets with transmission constraints but their models are not multi-stage.

\cite{Sauma2006} evaluate social welfare implications of investments in transmission capacity. Market deregulation has lead to independent companies that operate the transmission network but may not necessarily generate the electricity. For the generators action they use a simplified version of the \cite{Murphy2005} generation capacity investment model.

\cite{Barmack2007}

\cite{Centeno2007} model medium-term strategic generation decisions based on a conjectured price-response market equilibrium.

\cite{Genc2007}

\cite{Kiesling2007}





%%% Local Variables: 
%%% mode: latex
%%% TeX-master: "../emarket_simulation"
%%% End: 
