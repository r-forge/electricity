\section{Introduction}

In this paper we will build a model for generation capacity investment under uncertainty. We consider a deregulated electricity market, where typically several large players act on market leading to an oligopoly situation. There is not much literature which directly deals with the question of  generation capacity investments. Recent works are \cite{Chuang2001}, \cite{Ventosa2002}, \cite{Chaton2003}, \cite{Hogendorn2003}, \cite{Pineau2003}, \cite{Ehrenmann2004}. \cite{Murphy2005}, \cite{Kiesling2007}, \cite{Pineau2007}.

\subsection{Stochastic oligopoly models}

\textbf{References:} \cite{Haurie2002, Pineau2003}\\

$S$-adapted structure is similar to the open-loop case, except that the strategies of the players adapt to the sample path of the stochastic variable \citep[see][pg. 128]{Pineau2003}.

\cite{Haurie2002} developed an approximation method with variational inequalities for $S$-adapted oligopoly equilibria. It can be used with any discrete state process that can be represented as an event tree can be used as description of the random disturbances.



Existence of Nash-Cournot equilibria is important.

In theory, the output of a competitive generation market is equal to the output of a regulated system with a central planner that minimizes investment plus operating costs to meet demand (Green, 2000), see \citep[see][pg. 111]{Rothwell2003}.

Papers with focus investment problem: \cite{Pineau2003}, \cite{Murphy2005}, \cite{Genc2007}, \cite{Kiesling2007}, \cite{Barmack2007}, \cite{Sauma2006}

Market simulation: \cite{Torre2003}, \cite{Valenzuela2007}, \cite{Hobbs2001},\cite{Otero-Novas2000}

General review paper: \cite{Neuhoff2005}, \cite{Ventosa2005}, \cite{Kahn1998}

%%% Local Variables: 
%%% mode: latex
%%% TeX-master: "../emarket_simulation"
%%% End: 
