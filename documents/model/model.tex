\section{The model}

\subsection{An application of peak load pricing under imperfect competition}

We will now use a static version of our model to link it to the economic intuition just outlined and to show how decisive the impact of such a game theoretic situation on investment incentives actually is.

There are four players, indexed by $i\in N$ (RWE, EON, Vattenfall and EnBW) seven technologies $k\in K$  and six different states $m\in M$ in which the market might be. Each of the players maximizes the following profit function:

\begin{gather}
	\max \pi_i(q_{i,k}^m,K_{i,k},)= \sum_{m\in M} v_m\times\\ \nonumber  \left[(\alpha_m-\beta_m\sum_{i\in N} \sum_{k\in K} q_{i,k}^m ) \sum_{k\in K} q_{i,k}^m - \sum_{k\in K} c^k q_{i,k}^m \right] \\ \nonumber
			\text{s.t.:} \  q_{i,k}^m-K_{i,k} \leq 0;\  \forall i,k,m \\
 										  q_{i,k}^m	\geq 0; \ \forall i,k,m   \nonumber
\end{gather}

\begin{tabular}[c]{l l}
$i\in N$        & players, firms\\
$s\in S$       	& scenarios\\
$m\in M$	& states of the market \\
$k\in K$	& technologies \\
$t\in T$	& time \\
$K_{i,k}^t$      & available capacity at time $t$ from technology $k$ \\
                & for firm $i$ \\
$q_{i,k}^{s,m,t}$ & quantity at time $t$, technology $k$, firm $i$, \\
                & in market state $m$ and scenario $s$ \\			
$I_{i,k}^t$   & investment in technology $k$, at time $t$ from for firm $i$\\
$p_s$        & probability of scenario $s$\\
$\nu_m$      & says how often (how many hours) market \\
             & state $m$ occurs \\
$\epsilon$   & discount factor \\
$\alpha(m)$  & demand function intercept in market state $m$ \\
$\beta(m)$   & demand function slope in market state $m$ \\
$c_k$	     & variable costs of technology $k$	\\
$\Gamma_k$   & investment costs in technology $k$  \\
$F_k$        & scrap values  \\
\\
\end{tabular}

$\alpha(m)$ and $\beta(m)$ are the intercept and the slope of the inverse demand function in different market states which could be very high, high, intermediate etc.. $q_{i,k}^m$ is the quantity produced with each technology by each player in each market state and $K_i^k$ is the corresponding capacity. $K_i^t$ is now given for each player as this is the short run equilibrium. $c^k$ are the short run variable costs which are the same for every producer. The parameter $\nu_m $ says how often a certain demand state occurs. This parameter is used as a weight for the shadow price of capacity which we obtain. We derived the Karush Kuhn Tucker Conditions to obtain a Mixed Complementarity Problem and solved it by using the PATH Solver in GAMS.

Table \ref{tab:statquant} in the appendix shows the quantities predicted by our oligopoly model. While during a high demand state almost all capacities are used, during low demand states hard coal plants are the marginal technology. Prices differ significantly but are above marginal costs even if demand is very low. Please note that there is strategic capacity withholding as not all capacities are brought to the market even if more than the variable costs were covered.

As we used a Lagrangian method to calculate this short run equilibrium, we can have a look at the shadow prices of capacity to see how much an additional unit of capacity would be worth for every player. Please note that this values cannot be smaller than zero as, in our setup, there always exists the option of not using existing capacities. So, as in our example above, there are only positive scarcity rents if the capacity of the technology is binding. Table \ref{tab:statlambda} shows the shadow prices of capacity. The values can be read in such a way that an additional MW of hydro power capacity would increase profits RWE makes in all the exremely high market states by 831 EUROS and the overall yearly profit of RWE by 44832 EUROS. On the contrary, the same capacity is worth 269972 EUROS for EnBW. How can such different results be explained? First, it seems that firms gain less from investing during high demand states which is straightforward as demand is more inelastic during high demand states and so one looses more by adding more capacity which depresses the gain from capacity. On the other hand, there are positive values for almost all technologies during extremely high demand states as most of the capacities become binding then.
Most importantly, for smaller players like Vattenfall and EnBW shadow prices are far higher as the effect of less scarcity, and thereby lower prices has a smaller effect on their lower quantities sold. This means that it is impossible to assess the investment incentives a firm on a market like the electricity market (oligopoly and L-shaped cost functions) has without considering the strategic implications such investments have.

\subsection{Investments under uncertainty}

In this part, we introduce two deciding factors. First, dynamics and thereby investments which link the different time periods. Second, uncertainty which is accounted for by different scenarios that lead to a recourse problem. The uncertainty about future demand is accounted for by different demand scenarios. Our approach follows the game with expected scenarios (GPS) approach by \cite{Genc2007} which allows to address problems which are out of reach for dynamic programming. Our two stage Model looks as follows.

\begin{gather}
	\max \pi_i(q_{i,k}^{s,m,t},K^t_{i,k},I_{i,k})= \\ \nonumber \sum_m v_m \left[ (\alpha_m^l-\beta_m^l \sum_i\sum_k q_{i,k}^{l,m,1}) \sum_k q_{i,k}^{l,m,1} - \sum_k c_k q_{i,k}^{l,m,1} \right]  \\  \nonumber 
	+ \epsilon \sum_s p_s \sum_m v_m \left[ (\alpha_m^s-\beta_m^s \sum_i\sum_k q_{i,k}^{s,m,2}) \sum_k q_{i,k}^{s,m,2} - \sum_k c_k q_{i,k}^{s,m,2}  \right]   \\  \nonumber 
									- \sum_k \Gamma_k I_{i,k} + \epsilon \sum_k F_k I_{i,k}  \\       
			\text{s.t.:} \  q_{i,k}^{l,m,1} - K^{1}_{i,k} \leq 0; \ \forall i,k,m    \label{eq:oligopmax2}\\ 
											q_{i,k}^{s,m,2} - K^{2}_{i,k} \leq 0; \ \forall i,k,m,s  \label{eq:oligopmax3}\\
										  K^{2}_{i,k}  - \rho K^{1}_{i,k}  - I_{i,k} = 0 ; \ \forall i,k  \label{eq:ologopmax5} \\  
 										  q_{i,k}^{s,m,t};K^t_{i,k};I_{i,k}	\geq 0; \ \forall i,k,s   \nonumber
\end{gather}

Each player ($i$) maximizes its profit by setting $q_{i,k}^{s,m,t}$ and $I_{i,k}^t$ . By considering different demand developments and the associated probabilities ($p_s$) of the different scenarios ($s\in S$), the players take into account how demand might evolve in the future. Capacities now evolve over time $t\in T$, according to the state equation (\eqref{eq:ologopmax5}). \eqref{eq:oligopmax2} and \eqref{eq:oligopmax3} are the capacity constraints. Quantities are allowed to be adapted to different scenarios that evolve, thereby stating that firms can always react to demand by adjusting the short run production. On the contrary, investments ($I$) are not allowed to differ in such a way as they have to be set in advance when it is not clear jet how high demand might be. Please note, that quantities do not depend on what other players might invest. They do depend however, on how high own investments are. 
% as the cost function can be changed between period one and two($C^1_i vs. C^2_i$). (Idee - Wenn man das weggibt, m�sste man den Effekt den eigene Investments auf den Marktpreis haben sehen). 
If quantities would depend on Investments of other players as well, we would enter the realm of feedback or closed loop games (which are the same in the case of a two stage game). It has to be noted here that the solution of a closed loop game can, and will, in general, be different from the solution of an open loop game.

To solve the model, we derive the Karush Kuhn Tucker Conditions (KKT) to obtain a mixed complementarity problem (MCP) which we solve by using GAMS and the PATH solver. For the model above, we used the Cournot approach to derive the first order conditions. For the competitive benchmark we solved the problem under the assumption that just one, welfare maximizing player disposes of all the initial capacities of the four players. When deriving the KKT conditions, this player is assumed to set prices equal to marginal costs.
%\clearpage

