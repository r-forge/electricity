\section{The model}

Assumptions:

\begin{itemize}
	\item same discount rate on cashflows
\end{itemize}

The stochastic programming approach in \cite{Genc2007} makes it possible to compute $S$-adapted equilibra for much larger problems than is possible using dynamic programming.

\subsection{Notation}

\begin{longtable}[c]{l l}
$i\in N=\left\{ 1,2,\dots,n\right\}$                           & players, firms\\
$s\in S=\left\{ 1,2,\dots,\omega\right\}, \omega<\infty$       & scenarios\\
$p_s$                                                          & probability of scenario $s$\\
$k\in K=\left\{ 1,2,\dots,m\right\}$                           & available technologies\\
$K_{i,k}^t$                                                     & available capacity at time $t$ from technology $k$ for firm $i$\\
$K_i^t=\left[K_{i,1}^t,K_{i,2}^t,\dots,K_{i,m}^t\right]$                                    & total available capacity at time $t$ for firm $i$\\
$q_{i,s}^t\in\mathbb{R}_+^m$                                                     & quantity\\
$Q(K_t)=\sum_{i=1}^nq_i$                                        & total produced quantity of all firms\\
$P(Q)$                                                         & inverse demand function\\

$I_{i,s}^t$                                                     & investment quantities\\
$\Pi_i=\sum_{s=1}^\omega p_s\Pi_{i,s}$                            & total expected profit for player $i$
\end{longtable}

\subsection{Objective function}

\subsection{Constraints}

For formulation of cost functions see \cite{Bergman1995}, \cite{Pineau2003}.	

For existence proof of Nash-Cournot equilibria for oligopolies see \cite{Murphy1982}.



%%% Local Variables: 
%%% mode: latex
%%% TeX-master: "../emarket_simulation"
%%% End: 
