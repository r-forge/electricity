\section{Results} 

First, we consider a version of the model with two stages and two possible future demand states which are accounted for by a binomial tree structure. For the discount factor, we assume an interest rate of $5\%$ and thereby assume that this figure is the same for all players. Equivalently, we set the depreciation rate to $2.5\%$ a year. As we have two periods, and each is assumed to last for two years, this version of the model represents a medium term perspective. \\
In Table \ref{tab:quant}, quantities and market prices for time period one and the high demand state are shown. It can be seen that during the high demand state, all capacities are fully utilized.

\begin{table}[htb]
\centering
\caption{Equilibrium quantities at $t=0$ (high market state)}
\vspace{0.3cm}
% Table generated by Excel2LaTeX from sheet 'Tabelle1'
% Table generated by Excel2LaTeX from sheet 'Tabelle1'
\begin{tabular}{rrrrr}
\hline
           &        RWE &        EON & Vattenfall &       EnBW \\
\hline
     Hydro &        726 &       1304 &          9 &        438 \\

   Nuclear &       5478 &       8452 &       1412 &       4263 \\

     BCoal &      10531 &       1409 &       6923 &        444 \\

     HCoal &       7227 &       9439 &       1720 &       3279 \\

       Gas &       4277 &       3789 &        860 &       1074 \\

       Oil &        175 &       1763 &       1418 &        608 \\

      Pump &        778 &       1095 &       2870 &        358 \\
\hline
\end{tabular}  
\label{tab:quant}
\\
\vspace{0.3cm}
\scriptsize Source: own calculations
\end{table}

Table \ref{tab:lambda} shows the shadow price of capacity for each of the firms. It can be seen here, that the value of an additional unit of capacity is very different for different oligopolistic players. Actually it only takes on positive values for Vattenfall and EnBW during high and extremely high market states. For most of the players, additional capacity is of no value, even in high demand states. This is intuitive however, as an additional unit of capacity might be not worth it for a big player because it lowers the price he gets for all the other units sold. For small players like Vattenfall and EnbW an additional unit of capacity is valuable, at least in high market states.

\begin{table}[htb]
\centering
\caption{Shadow prices of an additional unit of capacity}
\vspace{0.3cm}
% Table generated by Excel2LaTeX from sheet 'Tabelle1'
% Table generated by Excel2LaTeX from sheet 'Tabelle1'
\begin{tabular}{lrrrr}
\hline
           & Vattenfall &            &       EnBW &            \\
\hline
           &    exthigh &      vhigh &    exthigh &      vhigh \\
\hline
     Hydro &     11.567 &          0 &     47.838 &     18.649 \\

   Nuclear &      9.601 &          0 &     45.886 &     16.365 \\

     BCoal &      8.479 &          0 &     44.838 &     15.648 \\

     HCoal &      2.996 &          0 &     39.289 &      9.811 \\

       Gas &          0 &          0 &     21.911 &          0 \\

       Oil &          0 &          0 &     11.429 &          0 \\

      Pump &          0 &          0 &          0 &          0 \\
\hline
\end{tabular}  
\label{tab:lambda}
\\
\vspace{0.3cm}
\scriptsize Source: own calculations
\end{table}

Having looked at the shadow prices it is now not surprising that EnBW and Vattenfall are the only two players which will invest. In Table \ref{tab:invest} the investment levels are given and are contrasted with the amounts of investments, a perfectly competitive market would yield. Apparently, there seems to be a distortion towards too low investments in the oligopoly case. It is also of interest, that a social planner would not invest in gas fired plants, suggesting that not only the level of investments, but also the technology mix used to satisfy the varying demands might be different in an oligopolistic market, than under perfect competition.

\begin{table}[htb]
\centering
\caption{Investments in different market forms}
\vspace{0.3cm}
% Table generated by Excel2LaTeX from sheet 'Tabelle1'
\begin{tabular}{llrrrr}
\hline
           &            &    Nuclear &      BCoal &      HCoal &        Gas \\

\hline
 Oligopoly &     Vatten &         13 &         91 &         48 &          2 \\

           &       EnBW &        329 &        399 &        319 &        150 \\

\hline
 Perfect comp. benchmark &            &        744 &       1387 &        910 &            \\
\hline
\end{tabular}  
\label{tab:invest}
\\
\vspace{0.3cm}
\scriptsize Source: own calculations
\end{table}

\section{Conclusions}

In this paper, we investigate inhowfar deregulated electricity markets can be expected to deliver optimal investments into plants. A purely analytical model cannot answer this question, as, from the analytical side, it is unclear whether the combination of a peak load pricing problem, an oligopolistic market structure and uncertainty will create under- or even overinvestment. The German electricity market provided us with a real world example for our numerical model. Building on \cite{Genc2007} we extend their framework further toward a more realistic representation of market states and technologies and develop a normative welfare-optimal benchmark. We came to the preliminary conclusion that there seems to be an underinvestment problem arising from the current market framework. Additionally, it seems as if also the chosen technology mix is distorted away from what is optimal. This conclusion looks even gloomier in light of aging plants which have to be replaced and the nuclear phase out.

Further research will focus on a model which allows more conclusions about the long run development of capacities. Additionally, the use of different information structures would be interesting.



%%% Local Variables: 
%%% mode: latex
%%% TeX-master: "../emarket_simulation"
%%% End: 
 
