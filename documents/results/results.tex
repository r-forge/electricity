\section{Results} 

In this first version of the model with only two stages we considered a framework in which firms decide upon investments now, and the second stage of the game lasts for five years. For the discount factor, we assume an interest rate of 3\% and thereby assume that this figure is the same for all players. Equivalently, we set the depreciation rate to 3\% a year. Our setup is such that we can answer to what extent, in the medium run, investment paths that arise from our oligopolistic market deviate from the optimum. In stage zero, the results are the same as in the static version of our model (table \ref{tab:statquant}). For the investment incentive, now expected shadow prices (by considering the different demand scenarios that might evolve) are considered by the firms when they decide whether and in which technology to invest in.

\begin{table}
\centering

% Table generated by Excel2LaTeX from sheet 'Tabelle1'
\begin{tabular}{llrrrrrr}
\hline
\hline
  Scenario &            &    exthigh &      vhigh &       high &      inter &        low &       vlow \\
\hline
       low &        RWE &      20437 &      20437 &      19700 &      16484 &      13657 &       8954 \\

           &        EON &      19959 &      18641 &      18641 &      16484 &      11483 &       8954 \\

           &     Vatten &      17788 &      16607 &      15834 &      15834 &      13657 &       8954 \\

           &       EnBW &      17738 &      17214 &      16293 &      16293 &      13499 &       8954 \\
\hline
           &  {\bf Sum} &      75922 &      72898 &      70468 &      65094 &      52295 &      35817 \\

           & {\bf price} &        186 &        132 &        102 &         72 &         45 &         26 \\


\hline

      high &        RWE &      21789 &      20437 &      20437 &      17923 &      14275 &       9465 \\

           &        EON &      21789 &      20214 &      18641 &      17923 &      12573 &       9465 \\

           &     Vatten &      17788 &      17788 &      16573 &      15834 &      14364 &       9465 \\

           &       EnBW &      17738 &      17738 &      17214 &      16293 &      13499 &       9465 \\
           \hline
           &  {\bf Sum} &      79105 &      76176 &      72864 &      67972 &      54711 &      37861 \\

           & {\bf prices:} &        200 &        140 &        112 &         77 &         48 &         26 \\
\hline
\hline
\end{tabular}  

\label{tab:dynquant}
\caption{quantities and prices for scenarios and market states}
\begin{center}
source: own calculations
\end{center}
\end{table}

\begin{table}
\centering

Total quantities and prices the firms offer in each market state and for each scenario are given in table \ref{tab:dynquant}. Of course, prices are lower in the low demand scenario than in the high demand scenario. However, in both cases, due to investments and thereby higher quantities, prices are lower than in period zero.
Compared to the optimal results for prices and qualities, which are given in table \ref{tab:dynquantopt}, prices are higher and quantities are lower.

Our main result are our predictions for invested quantities. It can be seen in table \ref{tab:invest} given the cost and demand information we have, brown coal seems to be the dominant technology choice. If we rule out brown coal as there is only a limited number of available sites, nuclear plants for the social planner, and nuclear and hard coal plants for the oligopolists become the technology of choice as can be seen in table \ref{tab:invest}.

\begin{tabular}{llrrrrrr}
\hline
\hline
           &            &    exthigh &      vhigh &       high &      inter &        low &       vlow \\
\hline
{\bf t = 0} &            &            &            &            &            &            &            \\
\hline
           &   quantity &      82498 &      82498 &      77344 &      73331 &      63273 &      41546 \\

           &     prices &        136 &         81 &         72 &         44 &         17 &         16 \\
\hline
{\bf t = 1} &            &            &            &            &            &            &            \\
\hline
       low &   quantity &      94509 &      91497 &      86167 &      81578 &      65952 &      44771 \\

           &     prices &         44 &         34 &         34 &         16 &         11 &         11 \\

           &            &            &            &            &            &            &            \\

      high &   quantity &      96839 &      94397 &      90860 &      84879 &      67284 &      47327 \\

           &     prices &         65 &         44 &         34 &         20 &         16 &         11 \\
\hline
\hline
\end{tabular}  
 

\label{tab:dynquantopt}
\caption{optimal quantities and prices for different times and scenarios}
\begin{center}
source: own calculations
\end{center}
\end{table}

What can be said without ambiguity, is that the amount of capacity which is invested in is considerably distorted downwards by the effects of oligopolistic competition in an otherwise unchanged model. For most of the players, additional capacity is of no value, even in high demand states. This is intuitive however, as an additional unit of capacity might be not worth it for a big player because it lowers the price he gets for all the other units sold. For small players like Vattenfall and EnbW an additional unit of capacity is of more value, at least in high market states. Additionally, is seems that imperfect competition distorts technology choices away from flexible but capital intensive investments.

\begin{table}
\centering

% Table generated by Excel2LaTeX from sheet 'Tabelle1'
\begin{tabular}{llrrrr}
\hline
\hline
Investments &            &            &            & \multicolumn{ 2}{r}{Scen. w. Brown Coal} \\
\hline
 Oligopoly &            & Brown Coal &            &    Nuclear &  Hard Coal \\
\hline
           &        EON &       1063 &            &            &            \\

           &     Vatten &       7256 &            &       6249 &        283 \\

           &       EnBW &       9102 &            &       8248 &            \\

           &        Sum &      17422 &            & \multicolumn{ 2}{c}{14780} \\
\hline
   Optimum &            &      31096 &            &      27658 &            \\
\hline
\hline
\end{tabular}  


\label{tab:invest}
\caption{investments in different market forms, with and without availability of B. Coal sites (MW)}
\begin{center}
source: own calculations
\end{center}
\end{table}

\section{Conclusions}

In this paper, we investigate inhowfar deregulated electricity markets can be expected to deliver optimal investments into plants. A purely analytical model cannot answer this question, as, from the analytical side, it is unclear whether the combination of a peak load pricing problem, an oligopolistic market structure and uncertainty will create under- or even overinvestment. The German electricity market provided us with a real world example for our numerical model. Building on \cite{Genc2007} we extend their framework further toward a more realistic representation of market states and technologies and develop a normative welfare-optimal benchmark. We came to the preliminary conclusion that there seems to be an underinvestment problem arising from the current market framework. Additionally, it seems as if also the chosen technology mix is distorted away from what is optimal. This conclusion looks even gloomier in light of aging plants which have to be replaced and the nuclear phase out.

Further research will focus on a model which allows more conclusions about the long run development of capacities. Additionally, the use of different information structures would be interesting.



%%% Local Variables: 
%%% mode: latex
%%% TeX-master: "../emarket_simulation"
%%% End: 
 
