\section{Results} 

In this first version of the model with only two stages we considered a framework in which firms decide upon investments now, and the second stage of the game lasts for five years. For the discount factor, we assume an interest rate of 3\% p.a. and thereby assume that this figure is the same for all players. Equivalently, we set the depreciation rate to 3\% per year. Our model setup is such that we can answer to what extent, in the medium run, investment paths that arise from our oligopolistic market deviate from the optimum. In the first stage, the results are the same as in the static version of our model (see Table \ref{tab:statquant}). For the investment incentives, now expected shadow prices (by considering the different demand scenarios that might evolve) are considered by the firms when they decide on whether and in which technology to invest in.

\begin{table}
\centering
\caption{Comparison of optimal vs. oligopoly results in different times and scenarios}
% Table generated by Excel2LaTeX from sheet 'Tabelle1'
% Table generated by Excel2LaTeX from sheet 'Tabelle1'
% Table generated by Excel2LaTeX from sheet 'Tabelle1'
% Table generated by Excel2LaTeX from sheet 'Tabelle1'
\begin{tabular}{llrrrrrr}
\hline
\hline
           &            &    exthigh &      vhigh &       high &      inter &        low &       vlow \\
\hline
     t = 0 &            &            &            &            &            &            &            \\
\hline
 olig. &          q &     72,863 &     69,580 &     66,698 &     61,660 &     49,427 &     34,847 \\

           &          p &        210 &        149 &        118 &         83 &         52 &         27 \\
\hline
   opt. &          q &     82,498 &     82,498 &     77,344 &     73,331 &     63,273 &     41,546 \\

           &          p &        136 &         81 &         72 &         44 &         17 &         16 \\
\hline
\hline
     t = 1 &            &            &            &            &            &            &            \\

 olig. &            &            &            &            &            &            &            \\

       low &          q &     75,922 &     72,898 &     70,468 &     65,094 &     52,295 &     35,817 \\

           &          p &        186 &        132 &        102 &         72 &         45 &         26 \\

      high &          q &     79,105 &     76,176 &     72,864 &     67,972 &     54,711 &     37,861 \\

           &          p &        200 &        140 &        112 &         77 &         48 &         26 \\
\hline
   opt. &            &            &            &            &            &            &            \\
       low &          q &     94,509 &     91,497 &     86,167 &     81,578 &     65,952 &     44,771 \\
           &          p &         44 &         34 &         34 &         16 &         11 &         11 \\
           &            &            &            &            &            &            &            \\
      high &          q &     96,839 &     94,397 &     90,860 &     84,879 &     67,284 &     47,327 \\
           &          p &         65 &         44 &         34 &         20 &         16 &         11 \\
\hline
\hline
\end{tabular}  

\label{tab:dynquant}
\begin{center}
source: own calculations
\end{center}
\end{table}

Total quantities and prices for the oligopoly model and the perfectly competitive model at each point in time, and each scenario are given in Table \ref{tab:dynquant}. Of course, prices are lower in the low demand scenarios than in the high demand scenarios. However, in all cases, due to investments and thereby higher quantities, prices in period two are lower than in period one.
Compared to the optimal results for prices and qualities, prices are always higher and quantities are lower in the oligopoly case.

Our main results are our predictions for invested quantities. It can be seen in Table \ref{tab:invest} that, given the cost and demand information we have, brown coal seems to be the dominant technology choice. If we rule out brown coal, which might be plausible as there is only a limited number of available sites, nuclear plants for the social planner, and nuclear and hard coal plants for the oligopolists become the technology of choice.

What can be said without ambiguity, is that the amount of capacity which is built is considerably distorted downwards by the effects of oligopolistic competition in an otherwise unchanged model. For most of the players, additional capacity is of no value, even in high demand states. This is intuitive however, as an additional unit of capacity might be not worth it for a big player because it lowers the price he gets for all the other units sold. For small players like Vattenfall and EnbW an additional unit of capacity is of more value, at least in high market states. Additionally, as Hard Coal is only a profitable investment option in an oligopoly it seems that imperfect competition distorts investment choices away from flexible but capital intensive technologies.

\begin{table}
\centering
\caption{Investments in different market forms, with and without availability of B. Coal sites (MW)}
% Table generated by Excel2LaTeX from sheet 'Tabelle1'
\begin{tabular}{llrrrr}
\hline
\hline
investments &            &            &            & \multicolumn{ 2}{r}{Scen. w. Brown Coal} \\
\hline
 oligopoly &            & Brown Coal &            &    Nuclear &  Hard Coal \\
\hline
           &        EON &       1,063 &            &            &            \\
           &     Vatten &       7,256 &            &       6,249 &        283 \\
           &       EnBW &       9,102 &            &       8,248 &            \\
           &        Sum &      17,422 &            & \multicolumn{ 2}{c}{14,780} \\
\hline
   optimum &            &      31,096 &            &      27,658 &            \\
\hline
\hline
\end{tabular}  

\label{tab:invest}
\begin{center}
source: own calculations
\end{center}
\end{table}

\section{Conclusions}

Our results suggest that it is impossible to assess the investment incentives for firms on an electricity market (oligopoly with $L$-shaped cost functions) without considering the strategic implications such investments have.

In this paper, we investigate inhowfar deregulated electricity markets can be expected to deliver optimal capacity investments. A purely analytical model cannot answer this question. As pointed out in Section \ref{sect:3}, it is unclear whether the combination of a peak load pricing problem in an oligopolistic market structure and uncertainty will create under- or even overinvestment. The German electricity market provided us with a real world example for our numerical model. Building on \cite{Genc2007} we extend their framework further toward a more realistic representation of market states and technologies and develop a normative welfare-optimal benchmark. Furthermore, we linked the interpretation of such a model to the economic literature. We came to the preliminary conclusion that there seems to be an underinvestment problem arising from the current market framework and that introducing more competition would increase investments. Additionally, it seems as if the current market setup distorts investment choices away from flexible but capital intensive technologies. This conclusion looks even gloomier in the light of aging plants which have to be replaced and the nuclear phase out.

Price Caps might not be as bad for investments as suggested by some authors as in the case of an oligopolistic investment model, they might even provide a remedy against underinvestments.

Further research will focus on a model which allows more conclusions about the long run development of capacities. Additionally, the use of different information structures seems promising.


 

%%% Local Variables: 
%%% mode: latex
%%% TeX-master: "../eem08"
%%% End: 
