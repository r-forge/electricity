\begin{landscape}

\begin{table}
\scriptsize
\begin{tabular}[h]{ccccc}
\hline
\textbf{Investment problem}\\
\hline
  Authors & Information structure & Solution method & Transmission network & Numerical application \\
\hline
\cite{Genc2007} & $S$-adapted open-loop & MCP & No & electricity market, Ontario, Canada\\
1 & 2 & 3 & 4 & 5 \\
\hline
\textbf{Stochastic oligopoly models}\\
\hline
\cite{Salant1982}\\
\cite{Wolf1997}\\
\cite{Haurie2001}\\
\cite{Haurie2002}\\
\cite{Murto2004}
\end{tabular}
\end{table}
bla ble blu


\textbf{References:} \cite{Salant1982, Wolf1997, Haurie2001, Haurie2002, Pineau2003, Murto2004}\\


The $S$-adapted information structure was introduced by \cite{Haurie1990}.
$S$-adapted structure is similar to the open-loop case, except that the strategies of the players adapt to the sample path of the stochastic variable \citep[see][pg. 128]{Pineau2003}.

\cite{Haurie2002} developed an approximation method with variational inequalities for $S$-adapted oligopoly equilibria. It can be used with any discrete state process that can be represented as an event tree can be used as description of the random disturbances.

\cite{Murto2004} solves the game with feedback information structure.

\cite{Haurie2001}, \cite{Genc2007}

developed an approximation method with variational inequalities for $S$-adapted oligopoly equilibria. It can be used with any discrete state process that can be represented as an event tree can be used as description of the random disturbances.


Market simulation: \cite{Torre2003}, \cite{Valenzuela2007}, \cite{Hobbs2001},\cite{Otero-Novas2000}

General review paper: \cite{Neuhoff2005}, \cite{Ventosa2005}, \cite{Kahn1998}

\end{landscape}



%%% Local Variables: 
%%% mode: latex
%%% TeX-master: "../emarket_simulation"
%%% End: 
