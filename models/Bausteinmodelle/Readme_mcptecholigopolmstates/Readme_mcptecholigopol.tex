 \documentclass[a4paper,12pt]{article}
\usepackage[]{german}
\usepackage{setspace}
\usepackage[latin1]{inputenc}
\usepackage{mathptmx}
\usepackage{graphicx}
\usepackage{natbib}
\usepackage[ps2pdf,%
 linktocpage,%
 colorlinks,%
 bookmarks,%
 bookmarksopen,%
 bookmarksnumbered]{hyperref}
\usepackage{amssymb,amsthm,amsmath,exscale}
\usepackage{float}
\usepackage{enumerate}
%\usepackage{ccaption}

%\newtheorem{definition}{Definition}
%\newtheorem{theorem}{Theorem}
\allowdisplaybreaks

\newtheorem*{lem}{Lemma} 
\newenvironment{pro}{\begin{proof}[Proof]}{\end{proof}}
\newtheorem{con}{Conjecture}

\theoremstyle{remark}
\newtheorem*{rem}{Remark} 

%The Role of Incentives and Strategic Behavior in Electricity and Quality Regulation

\newcommand{\Titel}{Research Proposal: \\ Die Rolle von Anreizen und strategischem Verhalten in der Strom- und Qualitätsregulierung } 
\newcommand{\Datum}{\today}
\newcommand{\AutorEins}{ Mag. Anton Burger 0051357 \\  }
\newcommand{\AutorZwei}{}
\newcommand{\AutorEinsEmail}{anton.burger@wu-wien.ac.at}
\newcommand{\AutorZweiEmail}{}
\newcommand{\AutorEinsTel}{+43 1 31336 5898}
\newcommand{\AutorZweiTel}{+43 1 31336 5899}
\newcommand{\WPNO}{4}

\pagestyle{headings}
\pagenumbering{roman}

\onehalfspacing
%\setlength{\parindent}{10pt}
%\parindent=\parindent 

% %---------------------DECKBLÄTTER ANFANG----------------------%
 \begin{document}


  \section{Intro}

There are four players, indexed by $i$ (RWE, EON, Vattenfal and EnBW) seven technologies indexed by $k$ and six different states $m$ in which the market might be. Each of the players maximizes the following profit function:

\begin{gather}
	\max_{q_{i,k}^m} \pi_i(q_{i,k}^m,K_{i,k},)= \sum_m v_m * \left[(\alpha_m-\beta_m\sum_i \sum_k q_{i,k}^m ) \sum_k q_{i,k}^m - \sum_k c^k q_{i,k}^m \right] \\
			\text{s.t.:} \  q_{i,k}^m-K_{i,k} \leq 0;\  \forall i,k,m \\  \nonumber
 										  q_{i,k}^m	\geq 0; \ \forall i,k,m   \nonumber
\end{gather}

$\alpha$ and $\beta$ are the intercept and the slope of the inverse demand function, $q_{i,k}^m$ is the quantity produced with each technology by each player in each market state and $K_i^k$ is the corresponding capacity. $K_i^t$ is given for each player as this is the short run equilibrium. $c^k$ are the short run variable costs of producers.

The Lagrangian of this problem looks as follows:

\begin{gather}
	L_i(q,\lambda,\mu)= \sum_m v_m * \left[ (\alpha(m)-\beta(m)\sum_i \sum_k q_{i,k}^m ) \sum_k q_{i,k}^m - \sum_k c^k q_{i,k}^m \right] - \lambda_i^k (q_{i,k}^m - K_i^k)-\mu_i^k(-q_{i,k}^m)
\end{gather}

Which has the following KKT Conditions. 

\begin{gather}
\frac{\partial L_i(q,\lambda,\mu)}{\partial q_{i,k}^m}	= v_m\left[ \alpha(m) - \beta(m) \sum_k q_{i,k}^m - \beta(m)\sum_i \sum_k q_{i,k}^m - c_i^k \right] - \lambda_{i,k}^m + \mu_i^k = 0 \ \forall i,k,m
\end{gather}

This means that we use the derivation of $L_i^k(\cdot)$ with respect to the quantities of each player and each technology in each market state. In our case, that´s 4*7*6=168 KKT conditions. This can, by combining it with the conditions that arise from the nonnegativity constraint, be simplified to:

\begin{gather}
\frac{\partial L_i(q,\lambda,\mu)}{\partial \lambda_{i,k}^m}	= v_m \left[ \alpha(m) - \beta(m) \sum_k q_{i,k}^m - \beta\sum_i \sum_k q_{i,k}^m - c^k \right] - \lambda_{i,k}^m \leq 0 \ \forall i,k,m
\end{gather}

With

\begin{gather}
q_{i,k}^m \geq 0 \ \forall i,k,m
\end{gather}

as it´s complement.
The other KKT conditions are:

\begin{gather}
\frac{\partial L_i^k(q,\lambda,\mu)}{\partial \lambda_i^k}	= q_{i,k}^m-K_i^k \leq 0 \ \forall i,k,m
\end{gather}

With 

\begin{gather}
\lambda_i^k \geq 0 \  \forall i,k,m
\end{gather}

as it´s complement.
\end{document}