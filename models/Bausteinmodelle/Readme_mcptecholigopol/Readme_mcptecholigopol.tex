\documentclass[a4paper,12pt]{article}
\usepackage[]{german}
\usepackage{setspace}
\usepackage[latin1]{inputenc}
\usepackage{mathptmx}
\usepackage{graphicx}
\usepackage{natbib}
\usepackage[ps2pdf,%
 linktocpage,%
 colorlinks,%
 bookmarks,%
 bookmarksopen,%
 bookmarksnumbered]{hyperref}
\usepackage{amssymb,amsthm,amsmath,exscale}
\usepackage{float}
\usepackage{enumerate}
%\usepackage{ccaption}

%\newtheorem{definition}{Definition}
%\newtheorem{theorem}{Theorem}
\allowdisplaybreaks

\newtheorem*{lem}{Lemma} 
\newenvironment{pro}{\begin{proof}[Proof]}{\end{proof}}
\newtheorem{con}{Conjecture}

\theoremstyle{remark}
\newtheorem*{rem}{Remark} 

%The Role of Incentives and Strategic Behavior in Electricity and Quality Regulation

\newcommand{\Titel}{Research Proposal: \\ Die Rolle von Anreizen und strategischem Verhalten in der Strom- und Qualit�tsregulierung } 
\newcommand{\Datum}{\today}
\newcommand{\AutorEins}{ Mag. Anton Burger 0051357 \\  }
\newcommand{\AutorZwei}{}
\newcommand{\AutorEinsEmail}{anton.burger@wu-wien.ac.at}
\newcommand{\AutorZweiEmail}{}
\newcommand{\AutorEinsTel}{+43 1 31336 5898}
\newcommand{\AutorZweiTel}{+43 1 31336 5899}
\newcommand{\WPNO}{4}

\pagestyle{headings}
\pagenumbering{roman}

\onehalfspacing
%\setlength{\parindent}{10pt}
%\parindent=\parindent 

% %---------------------DECKBLÄTTER ANFANG----------------------%
 \begin{document}


  \section{Intro}

There are four players, indexed by $i$ (RWE, EON, Vattenfal and EnBW) and seven technologies indexed by $k$. Each of the players maximizes the following profit function:

\begin{gather}
	\max_{q_i} \pi(q_i,K_i,)= (\alpha-\beta\sum_i \sum_k q_i^k ) \sum_k q_i^k - \sum_k c^k q_i^k  \\
			\text{s.t.:} \  q_i^k-K_i^k \leq 0; \\  \nonumber
 										  q_i^k	\geq 0  \nonumber
\end{gather}

$\alpha$ and $\beta$ are the intercept and the slope of the inverse demand function, $q_i^k$ is the quantity produced with each technology by each player and $K_i^k$ is the corresponding capacity. $K_i^t$ is given for each player as this is the short run equilibrium and $c^k$ are the short run variable costs of producers.

The Lagrangian of this problem looks as follows:

\begin{gather}
	L_i^k(q,\lambda,\mu)= (\alpha-\beta\sum_i \sum_k q_i^k ) \sum_k q_i^k - \sum_k c^k q_i^k - \lambda_i^k (q_i^k-K_i^k)-\mu_i^k(-q_i^k)
\end{gather}

Which has the following KKT Conditions. 

\begin{gather}
\frac{\partial L_i^k(q,\lambda,\mu)}{\partial q_i^k}	= \alpha - \beta \sum_k q_i^k - \beta\sum_i \sum_k q_i^k - c_i^k - \lambda_i^k + \mu_i^k = 0 \ \forall i,k
\end{gather}

This means that we use the derivation of $L_i^k(\cdot)$ with respect to the quantities of each player and each technology. In our case, that�s 4*7=28 KKT conditions. This can, by combining it with the conditions that arise from the nonnegativity constraint, be simplified to:

\begin{gather}
\frac{\partial L_i^k(q,\lambda,\mu)}{\partial \lambda_i^k}	= \alpha - \beta \sum_k q_i^k - \beta\sum_i \sum_k q_i^k - c_i^k - \lambda_i^k \leq 0 \ \forall i,k
\end{gather}

With

\begin{gather}
q_i^k \geq 0 \ \forall i,k
\end{gather}

as it�s complement.
The other KKT conditions are:

\begin{gather}
\frac{\partial L_i^k(q,\lambda,\mu)}{\partial \lambda_i^k}	= q_i^k-K_i^k \leq 0 \ \forall i,k
\end{gather}

With 

\begin{gather}
\lambda_i^k \geq 0 \  \forall i,k
\end{gather}

as it�s complement.
\end{document}