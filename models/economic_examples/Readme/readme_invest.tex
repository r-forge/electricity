 \documentclass[a4paper,12pt]{article}
\usepackage[]{german}
\usepackage{setspace}
\usepackage[latin1]{inputenc}
\usepackage{mathptmx}
\usepackage{graphicx}
\usepackage{natbib}
\usepackage[ps2pdf,%
 linktocpage,%
 colorlinks,%
 bookmarks,%
 bookmarksopen,%
 bookmarksnumbered]{hyperref}
\usepackage{amssymb,amsthm,amsmath,exscale}
\usepackage{float}
\usepackage{enumerate}
%\usepackage{ccaption}

%\newtheorem{definition}{Definition}
%\newtheorem{theorem}{Theorem}
\allowdisplaybreaks

\newtheorem*{lem}{Lemma} 
\newenvironment{pro}{\begin{proof}[Proof]}{\end{proof}}
\newtheorem{con}{Conjecture}

\theoremstyle{remark}
\newtheorem*{rem}{Remark} 

%The Role of Incentives and Strategic Behavior in Electricity and Quality Regulation

\newcommand{\Titel}{Research Proposal: \\ Die Rolle von Anreizen und strategischem Verhalten in der Strom- und Qualitätsregulierung } 
\newcommand{\Datum}{\today}
\newcommand{\AutorEins}{ Mag. Anton Burger 0051357 \\  }
\newcommand{\AutorZwei}{}
\newcommand{\AutorEinsEmail}{anton.burger@wu-wien.ac.at}
\newcommand{\AutorZweiEmail}{}
\newcommand{\AutorEinsTel}{+43 1 31336 5898}
\newcommand{\AutorZweiTel}{+43 1 31336 5899}
\newcommand{\WPNO}{4}

\pagestyle{headings}
\pagenumbering{roman}

\onehalfspacing
%\setlength{\parindent}{10pt}
%\parindent=\parindent 

% %---------------------DECKBLÄTTER ANFANG----------------------%
 \begin{document}


  \section{Intro}

This note demonstrates a simple model to illustrate how firms which are on a market where the logic of peak load pricing has to be applied, would behave if there is imperfect competition.

\begin{tabular}[c]{l l}
$i\in N$        & players, firms\\
$s\in S$       	& scenarios\\
%$m\in M$	& states of the market \\
%$k\in K$	& technologies \\
$t\in 0,1$	& time \\
$K_{i}$      & available capacity for firm $i$ \\
$q_{i}^{s,t}$ & quantity firm $i$, scenario $s$ \\			
$I_{i}$   & investment for firm $i$\\
$\rho_s$        & probability of scenario $s$\\
%$\nu_m$      & says how often (how many hours) market \\
%             & state $m$ occurs \\
%$\epsilon$   & discount factor \\
$\alpha^s$  & demand function intercept in scenario $s$ \\
$\beta$   & demand function slope \\
$c$	     & variable costs of technology $k$	\\
$\Gamma$   & investment costs \\
$p^{s,t}$   & market price 
\end{tabular}

\begin{gather}
	\max \pi_i(q_{i}^{s,t},K_{i},I_{i}^t)=
	\sum_s \rho_s \left[ (\alpha^s- \beta \sum_i q_{i}^{s,1}) q_{i}^{s,1} - c q_{i}^{s,1}  \right] - \Gamma I_{i}^{0}  \\       
			\text{s.t.:} \  q_{i}^{s,1} - K_{i}^1 \leq 0; \ \forall i,s  \label{eq:oligopmax2}\\ 
										  K^{1}_{i}  - K^{0}_{i}  - I_{i}^0 = 0 ; \ \forall i  \label{eq:ologopmax5} \\
										       \alpha^s - \beta \sum_i q_i^{s,1} - PC^1 \leq 0; \forall s \\ \nonumber
										  - p_s \left[ PC q_{i}^{s,1} - c q_{i}^{s,1} \right] + \Gamma I_{i} \leq 0; \forall i \\ \nonumber 
 										  q_{i}^{s};K^t_{i};I_{i}	\geq 0; \ \forall i,s  \nonumber
\end{gather}

\clearpage

This leads to the following Lagrangian / Hamiltonian:

\begin{gather}
	\max L_i(q_{i}^{s},K^t_{i},I_{i},\lambda_{i}^{s},u_{i}, \psi_s, \iota_i)= 
	 \sum_s \rho_s \left[(\alpha^s- \beta \sum_i q_{i}^{s,1}) q_{i}^{s,1} - c q_{i}^{s,1}  \right]	- \Gamma I_{i}  \\ \nonumber  
	  	- \lambda_{i}^{s}(q_{i}^{s} - K^{1}_{i}) \\ \nonumber
			- u_{i}(K^{1}_{i}  - K^{0}_{i}  - I_{i})		\\   \nonumber
			- \psi_s (\alpha^s - \beta  \sum_i q_i^{s,1} - PC ) \\ \nonumber
			-\iota_i (-\sum_s p_s \left[ PC q_{i}^{s,1} - c q_{i}^{s,1} \right] + \Gamma I_{i})	\nonumber			
\end{gather}

and the following KKT Conditions:
\begin{gather}
 - \rho_s \left[ \alpha^s - \beta q_{i}^{s} - \beta \sum_i q_{i}^{s} - c \right] + \lambda_{i}^{s} - \beta \psi_s - \iota_i p_s (PC-c) \geq 0 \\ \nonumber \ \bot \ q_{i}^{s,t} \geq 0;\ \forall i,s \\
\frac{\partial L(\cdot)}{\partial \lambda_{i}^{s}} = - q_{i}^{s} + K^t_{i} \geq 0 \ \bot \ \lambda_{i}^{s} \geq 0 ; \\
\frac{\partial L(\cdot)}{\partial I_{i,k}} =  \Gamma - u_{i} + \iota_{i} \Gamma \geq 0 \ \bot \ I_{i} \geq 0 ;\forall i \\
\frac{\partial L(\cdot)}{\partial u} = K^1_{i} -  K^0_{i} - I_{i}  = 0 \ \bot \ u_{i} \ \mbox{\textit{free}}; \  \forall i  \\
\frac{\partial L(\cdot)}{\partial K^1_{i}} = -\sum_s  \lambda_{i}^{s} + u_{i} \geq 0  \ \bot \ K^1_{i} \geq 0 ; \forall i,k \\
\frac{\partial L(\cdot)}{\partial \psi_s} = -\alpha^s + \beta  \sum_i q_i^{s} + PC   \geq 0  \ \bot \ \psi_s \geq 0 ; \forall s \\
\frac{\partial L(\cdot)}{\partial \iota_i} = - \sum_s p_s \left[ (min(PC,MP) - c) q_{i}^{s,1} \right] + \Gamma I_{i})  \geq 0  \ \bot \ \iota_i \geq 0 ; \forall i \\
\end{gather}

in the first kkt it can be seen that, as q is bigger than zero in every sensible case, the first condition must fulfilled with equality. By rearranging, ass�ming the monopoly case and setting jota to zero (so we assume that the price cap is binding, but does not endanger the financial viability of the firm) we get:

\begin{gather}
    \rho_s c + \lambda_{i}^{s}  = rho*( \alpha^s - 2*\beta q_{i}^{s}) + \beta \psi_s  \\ \nonumber
\end{gather}

Which means that expected marginal revenue must be equal to marginal costs plus lambda, the scarcity rent. If the price cap $PC^1$ is binding, $\psi_s$ increases, if the price cap is lowered. Under imperfect competition, prices are distorted away from competitive levels, and q is set too low. At least in theory, $\psi_s$ can be set high enough to offset the distortion coming from imperfect competition. In other words - a price cap distorts the quantity choice on a monopolist or an oligopolist upwards.

Under the assumption of perfect Competition, the first FOC looks as follows:
\begin{gather}
\frac{\partial L(\cdot)}{\partial q_{i}^{s,t}} = p_s \left[ \alpha^s -  \beta \sum_i q_{i}^{s,t} - c \right] - \lambda_{i}^{s,t} - \beta \psi_s \leq 0 \\ \nonumber \ \bot \ q_{i}^{s,t} \geq 0;\ \forall i,s \\
\end{gather}


\end{document}