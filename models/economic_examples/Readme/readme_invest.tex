 \documentclass[a4paper,12pt]{article}
\usepackage[]{german}
\usepackage{setspace}
\usepackage[latin1]{inputenc}
\usepackage{mathptmx}
\usepackage{graphicx}
\usepackage{natbib}
\usepackage[ps2pdf,%
 linktocpage,%
 colorlinks,%
 bookmarks,%
 bookmarksopen,%
 bookmarksnumbered]{hyperref}
\usepackage{amssymb,amsthm,amsmath,exscale}
\usepackage{float}
\usepackage{enumerate}
%\usepackage{ccaption}

%\newtheorem{definition}{Definition}
%\newtheorem{theorem}{Theorem}
\allowdisplaybreaks

\newtheorem*{lem}{Lemma} 
\newenvironment{pro}{\begin{proof}[Proof]}{\end{proof}}
\newtheorem{con}{Conjecture}

\theoremstyle{remark}
\newtheorem*{rem}{Remark} 

%The Role of Incentives and Strategic Behavior in Electricity and Quality Regulation

\newcommand{\Titel}{Research Proposal: \\ Die Rolle von Anreizen und strategischem Verhalten in der Strom- und Qualitätsregulierung } 
\newcommand{\Datum}{\today}
\newcommand{\AutorEins}{ Mag. Anton Burger 0051357 \\  }
\newcommand{\AutorZwei}{}
\newcommand{\AutorEinsEmail}{anton.burger@wu-wien.ac.at}
\newcommand{\AutorZweiEmail}{}
\newcommand{\AutorEinsTel}{+43 1 31336 5898}
\newcommand{\AutorZweiTel}{+43 1 31336 5899}
\newcommand{\WPNO}{4}

\pagestyle{headings}
\pagenumbering{roman}

\onehalfspacing
%\setlength{\parindent}{10pt}
%\parindent=\parindent 

% %---------------------DECKBLÄTTER ANFANG----------------------%
 \begin{document}


  \section{Intro}

This is a simple version of the model which shall demonstrate the basic economic effects (Please note that sums are written in a simplified way):

\begin{tabular}[c]{l l}
$i\in N$        & players, firms\\
$s\in S$       	& scenarios\\
%$m\in M$	& states of the market \\
%$k\in K$	& technologies \\
$t\in T$	& time \\
$K_{i}^t$      & available capacity at time $t$ from technology $k$ \\
                & for firm $i$ \\
$q_{i}^{s,t}$ & quantity at time $t$, technology $k$, firm $i$, \\
                & in market state $m$ and scenario $s$ \\			
$I_{i}^t$   & investment in technology $k$, at time $t$ from for firm $i$\\
$p_s$        & probability of scenario $s$\\
%$\nu_m$      & says how often (how many hours) market \\
%             & state $m$ occurs \\
%$\epsilon$   & discount factor \\
$\alpha$  & demand function intercept in market state $m$ \\
$\beta$   & demand function slope in market state $m$ \\
$c$	     & variable costs of technology $k$	\\
$\Gamma$   & investment costs in technology $k$  \\
$F$        & scrap values  \\
\end{tabular}

\begin{gather}
	\max \pi_i(q_{i}^{s,t},K^t_{i},I_{i})=  (\alpha^0-\beta \sum_i q_{i}^{0}) q_{i}^{0} - c q_{i}^{0}  \\  \nonumber 
	+ \sum_s p_s \left[ (\alpha^s- \beta \sum_i q_{i}^{s,1}) q_{i}^{s,1} - c q_{i}^{s,1}  \right]   \\  \nonumber 
									- \Gamma I_{i} +  F I_{i}  \\       
			\text{s.t.:} \  q_{i}^{s,t} - K^{t}_{i} \leq 0; \ \forall i,t,s  \label{eq:oligopmax2}\\ 
										  K^{2}_{i}  - K^{1}_{i}  - I_{i} = 0 ; \ \forall i  \label{eq:ologopmax5} \\  
										  \alpha^s - \beta \sum_i q_i^{s,1} - PC \leq 0; \forall s \\ \nonumber 
 										  q_{i}^{s,t};K^t_{i};I_{i}	\geq 0; \ \forall i,s,t   \nonumber
\end{gather}

This leads to the following Lagrangian / Hamiltonian:

\begin{gather}
	\max L_i(q_{i}^{s,t},K^t_{i},I_{i},\lambda_{i}^{s,t},u_{i}, \psi)= (\alpha^0-\beta \sum_i q_{i}^{0}) q_{i}^{0} - c q_{i}^{0}	+ \\  \nonumber 
	 \sum_s p_s \left[(\alpha^s- \beta \sum_i q_{i}^{s,1}) q_{i}^{s,1} - c q_{i}^{s,1}  \right] \\  \nonumber 
		- \Gamma I_{i} +  F I_{i}  \\ \nonumber  
	  	+ \lambda_{i}^{0}(q_{i}^{0} - K^{0}_{i})+ \lambda_{i}^{s,1}(q_{i}^{s,1} - K^{1}_{i}) \\ \nonumber
							u_{i}(K^{1}_{i}  - K^{0}_{i}  - I_{i})		\\   \nonumber
							\psi_s (\alpha^s - \beta  \sum_i q_i^{s,2} - PC )
\end{gather}

and the following KKT Conditions:

\begin{gather}
\frac{\partial L(\cdot)}{\partial q_{i}^{s,t}} = p_s \left[ \alpha^s - \beta q_{i}^{s,t} - \beta \sum_i q_{i}^{s,t} - c \right] - \lambda_{i}^{s,t} - \beta \psi_s \leq 0 \\ \nonumber \ \bot \ q_{i}^{s,t} \geq 0;\ \forall i,s \\
\frac{\partial L(\cdot)}{\partial \lambda_{i}^{s,t}} = q_{i}^{s,t} - K^t_{i} \leq 0 \ \bot \ \lambda_{i}^{s,t} \geq 0 ; \\
\frac{\partial L(\cdot)}{\partial I_{i,k}} = - \Gamma + \epsilon F + u_{i} \leq 0 \ \bot \ I_{i} \geq 0 ;\forall i \\
\frac{\partial L(\cdot)}{\partial u} = -K^1_{i} +  K^0_{i} - I_{i}  = 0 \ \bot \ u_{i} \ \mbox{\textit{free}}; \  \forall i  \\
\frac{\partial L(\cdot)}{\partial K^1_{i}} = \sum_s  \lambda_{i}^{s,1} - u_{i} \leq 0  \ \bot \ K^2_{i} \geq 0 ; \forall i,k
\end{gather}
Under the assumption of perfect Competition, the first FOC looks as follows:
\begin{gather}
\frac{\partial L(\cdot)}{\partial q_{i}^{s,t}} = p_s \left[ \alpha^s -  \beta \sum_i q_{i}^{s,t} - c \right] - \lambda_{i}^{s,t} - \beta \psi_s \leq 0 \\ \nonumber \ \bot \ q_{i}^{s,t} \geq 0;\ \forall i,s \\
\end{gather}



\end{document}