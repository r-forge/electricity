\section{Strategic Capacity Investments under Imperfect Competition}

\subsection{Open-Loop Cournot Model}

\begin{frame}

\frametitle{Open-Loop Cournot Model (Murphy and Smeers, 2005)}

\begin{beamerboxesrounded}
{Problem of generator $i$ in the \alert{short-run}}
\begin{gather}
	\min_{x_i,y_i^s} \sum_s \left[ - \alpha^s + \beta (y_i^s+y_{-i}^s) \right] y_i^s + v_i \sum_s y_i^s + K_i x_i \\
\text{s.t.:} \  x_i-y_i^s \geq 0; \  y_i^s \geq 0; \ \nonumber
\end{gather}
{\small
\begin{tabbing}
whereby: \= $y_i^s$ \  \= output per player and demand segment \\
\> $x_i$   \    \> capacity  \\
\> $\alpha - \beta y$    \   \= inverse demand curve \\
\> $v_i, K_i$    \    \> variable and capacity costs
\end{tabbing}}
\end{beamerboxesrounded}
\vspace{0.3cm}
\textbf{First order conditions:}
\begin{equation}
	\frac{\partial (- \pi)}{\partial y_i^s} = -\alpha^s + 2 \beta y_i^s +  y_{-i}^s + v_i + \lambda_i^s = \omega_i^s 
\end{equation}
By using the FOCs to obtain reaction functions, the \emph{mutual best responses} can be found.
\end{frame}

\subsection{Closed-Loop Cournot Model}

\begin{frame}
					
\frametitle{Closed-Loop Cournot Model (Murphy and Smeers, 2005)}

Quantities in the second period are a function of investments in the first period, i.e. the solution of the short-run equilibrium $y_i^s(x_i;x_{-i})$.

\vspace{0.3cm}
\begin{beamerboxesrounded}{Problem of generator $i$ in the \alert{long-run}}
\begin{equation}
	\min_{x_i} \sum_{s=1}^S \left[ - \alpha^s + \beta( y_i^s(x_i;x_{-i})+ y_{-i}^s(x_i;x_{-i})) + v_i \right] y_i^s(x_i;x_{-i}) + K_i x_i
\end{equation}
\end{beamerboxesrounded}
\vspace{0.3cm}

\textbf{First order conditions:}

\begin{equation}
	\frac{\partial (- \pi)}{\partial x_i} = \sum_s \left[-\alpha^s + 2 \beta y_i^s (\cdot) + \beta y_{-i}^s(\cdot) + v_i \right] \frac{\partial y_i^s (\cdot)}{\partial x_i} + \sum_s \beta y_i^s (\cdot) \frac{\partial y_{-i}^s (\cdot)}{\partial x_i} + K_i
\end{equation}

The effect of an investment of player $i$ on the action of player $-i$ in the second period $\partial y_{-i} (\cdot)/\partial x_i$ is now taken into account.


							\end{frame}