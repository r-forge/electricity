
\documentclass[mathserif,10pt]{beamer}
\hypersetup{pdfmode=FullScreen}
%\usepackage{pgfpages}
%\usepackage{qtree}

\mode<presentation>
{
  \usetheme{shadow}
  \usecolortheme{crane}
  \setbeamercovered{transparent}
  \useoutertheme{infolines}
}

\usepackage{times}
\usepackage{graphicx}
\usepackage[T1]{fontenc}
\usepackage{amssymb,amsmath,amsthm, amsbsy, bm}
\newcommand{\bs}{\boldsymbol}
\newcommand{\dd}{\mathrm{d}}
\beamertemplatenavigationsymbolsempty
\DeclareMathOperator{\maximize}{maximize}
\DeclareMathOperator{\E}{\mathbb{E}}
\DeclareMathOperator{\V}{V\mathbb{V}}
\DeclareMathOperator{\VaR}{VaR}
\DeclareMathOperator{\CVaR}{CVaR}

\title[Electricity Market Regulation] 
{An Investment Model of the Wholesale \\Electricity Market in Austria}

\author[Anton Burger\and Robert Ferstl]{Anton Burger\inst{1} \and Robert Ferstl\inst{2}}


\institute[]{
\inst{1}Institute for Operations Research\\
Vienna University of Economics and Business Administration
\and
\inst{2}Institute for Regulatory Economics\\
Vienna University of Economics and Business Administration}

\date[INFORMS]{\scriptsize\em \textbf{INFORMS International Conference,\\Puerto Rico\\\vspace{0.2cm} July 8-11, 2007}}

\pgfdeclareimage[height=1.0cm]{wu-logo}{wu-logo}
\logo{\pgfuseimage{wu-logo}}	% show logo

%\setbeameroption{show notes}
\setbeameroption{hide notes}

\begin{document}

\frame{\titlepage}

\section<presentation>*{Outline}

\begin{frame}
  \frametitle{Outline}
  \tableofcontents[pausesections]
\end{frame}

%\AtBeginSubsection[]
%{
%  \begin{frame}<beamer>
%    \frametitle{Outline}
%    \tableofcontents[current,currentsubsection]
%  \end{frame}
%}




\section{Motivation}

\begin{frame}
\frametitle{Motivation}
%\framesubtitle{}
\begin{itemize}
   \item \textcolor{craneblue}{\textbf{How can a company find an optimal cash management strategy including uncertainty?}}
    \begin{itemize}
        \item risk aversion of the company
        \item initial holdings in cash, bonds and stocks
         \item transaction costs
         \item \textcolor{alert}{\textbf{uncertainty}} of future short-term interest rate and equity returns
     \end{itemize}
\end{itemize}
\vspace{0.2cm}
\pause
\begin{beamerboxesrounded}[shadow=true]{Model features}
  $\Rightarrow$ \textcolor{craneblue}{\textbf{Multi-stage stochastic linear program}}
  \begin{itemize}
    \item based on bond portfolio management model, Dupacova and Bertocchi (2001), Bertocchi et al. (2006)
    \item possible assets are cash, bonds and stocks
    \item objective function minimizes a \textcolor{alert}{risk measure} ($\CVaR$)
    \item scenario generation for interest rates with short rate model, plus \textcolor{alert}{extension for equity returns}
    \item change of measure to real world with \textcolor{alert}{market price of risk} estimation
  \end{itemize}
\end{beamerboxesrounded}
\end{frame}


\section{References}
\scriptsize
%\tiny

\begin{frame}[allowframebreaks]
      \frametitle<presentation>{References}

      \begin{thebibliography}{8}

%      \beamertemplatebookbibitems
%        \bibitem{Birge1997}
%       John R. Birge, Francois Louveaux
%       \newblock Introduction to Stochastic Programming
%       \newblock {\em Springer}, 1997

    %\framebreak

    \beamertemplatearticlebibitems

    \bibitem{Genc2007}
    Talat S. Genc and Stanley S. Reynolds and Suvrajeet Sen
    \newblock Dynamic oligopolistic games under uncertainty: A stochastic programming approach
    \newblock {\em Journal of Economic Dynamics and Control}, 31(1):55-80, 2007 

    \bibitem{Murphy2005}
    Frederic H. Murphy and Yves Smeers
    \newblock Generation Capacity Expansion in Imperfectly Competitive Restructured Electricity Markets
    \newblock {\em Operations Research}, 53(4):646--661, 2005

   \end{thebibliography}
\end{frame}

\end{document}